%% USPSC-TCC-modelo-EESC.tex
% ---------------------------------------------------------------
% USPSC: Modelo de Trabalho Acadêmico (tese de doutorado, dissertacao de
% mestrado e trabalhos monograficos em geral) em conformidade com 
% ABNT NBR 14724:2011: Informação e documentação - Trabalhos acadêmicos -
% Apresentação
%----------------------------------------------------------------
%% Esta é uma customização do abntex2-modelo-trabalho-academico.tex de v-1.9.5 laurocesar 
%% para as Unidades do Campus USP de São Carlos:
%% EESC - Escola de Engenharia de São Carlos
%% IAU - Instituto de Arquitetura e Urbanismo
%% ICMC - Instituto de Ciências Matemáticas e de Computação
%% IFSC - Instituto de Física de São Carlos
%% IQSC - Instituto de Química de São Carlos
%%
%% Este trabalho utiliza a classe USPSC (USPSC.cls e USPSC1.cls) que é mantida 
%% pela seguinte equipe:
%% 
%% Coordenação e Programação:
%%   - Marilza Aparecida Rodrigues Tognetti - marilza@sc.usp.br (PUSP-SC)
%%   - Ana Paula Aparecida Calabrez - aninha@sc.usp.br (PUSP-SC)
%% Normalização:
%%   - Brianda de Oliveira Ordonho Sigolo - brianda@usp.br (IAU)
%%   - Eduardo Graziosi Silva - edu.gs@sc.usp.br (EESC)
%%   - Eliana de Cássia Aquareli Cordeiro - eliana@iqsc.usp.br (IQSC)
%%   - Flávia Helena Cassin - cassinp@sc.usp.br (EESC)
%%   - Maria Cristina Cavarette Dziabas - mcdziaba@ifsc.usp.br (IFSC)
%%   - Regina Célia Vidal Medeiros - rcvmat@icmc.usp.br (ICMC)
%%
%% Os modelos desenvolvidos utilizam diversos arquivos relacionado em 
%% 2.1 Pacote USPSC: Classe USPSC e modelos de trabalhos acadêmicos	do Tutorial do Pacote 
%%  USPSC para modelos de trabalhos de acadêmicos em LaTeX - versão 3.2

%----------------------------------------------------------------
%% Sobre a classe abntex2.cls:
%% abntex2.cls, v-1.9.5 laurocesar
%% Copyright 2012-2015 by abnTeX2 group at https://www.abntex.net.br/ 
%%
%----------------------------------------------------------------

\documentclass[
% -- opções da classe memoir --
12pt,		% tamanho da fonte
openright,	% capítulos começam em pág ímpar (insere página vazia caso preciso)
twoside,  % para impressão em anverso (frente) e verso. Oposto a oneside - Nota: utilizar \imprimirfolhaderosto*
%oneside, % para impressão em páginas separadas (somente anverso) -  Nota: utilizar \imprimirfolhaderosto
% inclua uma % antes do comando twoside e exclua a % antes do oneside 
a4paper,			% tamanho do papel. 
% -- opções da classe abntex2 --
chapter=TITLE,		% títulos de capítulos convertidos em letras maiúsculas
% -- opções do pacote babel --
english,			% idioma adicional para hifenização
french,				% idioma adicional para hifenização
spanish,			% idioma adicional para hifenização
brazil				% o último idioma é o principal do documento
% {USPSC-classe/USPSC} configura o cabeçalho contendo apenas o número da página
]{USPSC-classe/USPSC}
%]{USPSC-classe/USPSC1}
% Inclua % antes de ]{USPSC-classe/USPSC} e retire a % antes de %]{USPSC-classe/USPSC1} para utilizar o 
% cabeçalho diferenciado para as páginas pares e ímpares:
%- páginas ímpares: com seções ou subseções e o número da página
%- páginas pares: com o número da página e o título do capítulo 
% ---
% ---
% Pacotes básicos - Fundamentais 
% ---
\usepackage{xurl} % para quebra de linhas de urls
\usepackage[T1]{fontenc}		% Seleção de códigos de fonte.
\usepackage[utf8]{inputenc}		% Codificação do documento (conversão automática dos acentos)
\usepackage{lmodern}			% Usa a fonte Latin Modern
% Para utilizar a fonte Times New Roman, inclua uma % no início do comando acima  "\usepackage{lmodern}"
% Abaixo, tire a % antes do comando  \usepackage{times}
%\usepackage{times}		    	% Usa a fonte Times New Roman	
% Para usar a fonte , lembre-se de tirar a % do comando %\renewcommand{\ABNTEXchapterfont}{\rmfamily}, localizado mais abaixo, logo após "Outras opções para nota de rodapé no Sistema Numérico" 						
\usepackage{lastpage}			% Usado pela Ficha catalográfica
\usepackage{indentfirst}		% Indenta o primeiro parágrafo de cada seção.
\usepackage{color}				% Controle das cores
\usepackage{graphicx}			% Inclusão de gráficos
\usepackage{float} 				% Fixa tabelas e figuras no local exato
% \usepackage{chemfig}            % Para escrever reações químicas
% \usepackage{chemmacros}         % Para escrever reações químicas
\usepackage{tikz}				% Para escrever reações químicas e outros
\usetikzlibrary{positioning}
\usepackage{microtype} 			% para melhorias de justificação
\usepackage{pdfpages}
\usepackage{makeidx}            % para gerar índice remissivo
\usepackage{hyphenat}          % Pacote para retirar a hifenizacao do texto
\usepackage[absolute]{textpos} % Pacote permite o posicionamento do texto
\usepackage{eso-pic}           % Pacote para incluir imagem de fundo
\usepackage{makebox}           % Pacote para criar caixa de texto
% ---

% ---
% Pacotes de citações
% Citações padrão ABNT
% ---
% Sistemas de chamada: autor-data ou numérico.
% Sistema autor-data
\usepackage[alf, abnt-emphasize=bf, abnt-thesis-year=both, abnt-repeated-author-omit=no, abnt-last-names=abnt, abnt-etal-cite=3, abnt-etal-list=3, abnt-etal-text=it, abnt-and-type=e, abnt-doi=doi, abnt-url-package=none, abnt-verbatim-entry=no]{abntex2cite}
\bibliographystyle{USPSC-classe/abntex2-alf-USPSC}
% \bibliographystyle{abntex2-alf}
% Se o idioma for o inglês, inclua % no comando acima e exclua o % do comando abaixo
%\bibliographystyle{USPSC-classe/abntex2-alfeng-USPSC}

% Para o IQSC, que indica todos os autores nas referências, incluir % no início dos comandos acima e retirar a % dos comandos abaixo 
%\usepackage[alf, abnt-emphasize=bf, abnt-thesis-year=both, abnt-repeated-author-omit=no, abnt-last-names=abnt, abnt-etal-cite=3, abnt-etal-list=0, abnt-etal-text=it, abnt-and-type=e, abnt-doi=doi, abnt-url-package=none, abnt-verbatim-entry=no]{abntex2cite} 
%\bibliographystyle{USPSC-classe/abntex2-alf-USPSC}
% Se o idioma for o inglês, exclua % no comando acima ou do comando abaixo
%\bibliographystyle{USPSC-classe/abntex2-alfeng-USPSC}

% Sistema Numérico
% Para citações numéricas, sistema adotado pelo IFSC, incluir % no início dos comandos acima e retirar a % dos comandos abaixo 
%\usepackage{cite}              % agrupa citações numéricas consecutivas
%\usepackage[num, abnt-emphasize=bf, abnt-thesis-year=both, abnt-repeated-author-omit=no, abnt-last-names=abnt, abnt-etal-cite=3, abnt-etal-list=3, abnt-etal-text=it, abnt-and-type=e, abnt-doi=doi, abnt-url-package=none, abnt-verbatim-entry=no]{abntex2cite} 
%\bibliographystyle{USPSC-classe/abntex2-num-USPSC}
% Se o idioma for o inglês, exclua % no comando acima ou do comando abaixo
%\bibliographystyle{USPSC-classe/abntex2-numeng-USPSC}

% Complementarmente, verifique as instruções abaixo sobre os Pacotes de Nota de rodapé
% ---
% Pacotes de Nota de rodapé
% Configurações de nota de rodapé

% O presente modelo adota o formato numérico para as notas de rodapés quando utiliza o sistema de chamada autor-data para citações e referências. Para utilizar o sistema de chamada numérico para citações e referências, habilitar um dos comandos abaixo.
% Há diversa opções para nota de rodapé no Sistema Numérico.  Para o IFSC, habilitade o comando abaixo.

%\renewcommand{\thefootnote}{\fnsymbol{footnote}}  %Comando para inserção de símbolos em nota de rodapé

% Outras opções para nota de rodapé no Sistema Numérico:
%\renewcommand{\thefootnote}{\alph{footnote}}      %Comando para inserção de letras minúscula em nota de rodapé
%\renewcommand{\thefootnote}{\Alph{footnote}}      %Comando para inserção de letras maiúscula em nota de rodapé
%\renewcommand{\thefootnote}{\roman{footnote}}     %Comando para inserção de números romanos minúsculos  em nota de rodapé
%\renewcommand{\thefootnote}{\Roman{footnote}}     %Comando para inserção de números romanos minúsculos  em nota de rodapé

\renewcommand{\footnotesize}{\small} %Comando para diminuir a fonte das notas de rodapé
%Para utilizar a fonte Times New Roman, inclua retire % do início do comando abaixo 
%\renewcommand{\ABNTEXchapterfont}{\rmfamily}

% ---
% Pacote para agrupar a citação numérica consecutiva
% Quando for adotado o Sistema Numérico, a exemplo do IFSC, habilite 
% o pacote cite abaixo retirando a porcentagem antes do comando abaixo
%\usepackage[superscript]{cite}	

% ---
% Pacotes adicionais, usados apenas no âmbito do Modelo Canônico do abnteX2
% ---
\usepackage{lipsum}				% para geração de dummy text
% ---

% pacotes de tabelas
\usepackage{multicol}	% Suporte a mesclagens em colunas
\usepackage{multirow}	% Suporte a mesclagens em linhas
\usepackage{longtable}	% Tabelas com várias páginas
\usepackage{threeparttablex}    % notas no longtable
\usepackage{array}

% ----
% Compatibilização com a ABNT NBR 6023:2018 e 10520:2023
% Para tirar <> da URL e tornar as expressões latinas em itálico
\usepackage{USPSC-classe/ABNT6023-10520}
% As demais compatibilizações estão nos arquivos abntex2-alf-USPSC.bst,abntex2-alfeng-USPSC.bst, abntex2-num-USPSC.bst e abntex2-numeng-USPSC.bst, dependendo do idioma do textos e se o sistemas de chamada for autor-data ou numérico, conforme explicitado acima.
% ----

\usepackage{subcaption} % para subfigures
\DeclareCaptionSubType{quadro} 

% ---
% DADOS INICIAIS - Define sigla com título, área de concentração e opção do Programa 
% Consulte a tabela referente aos Programas, áreas e opções de sua unidade contante do
% arquivo USPSC-Siglas estabelecidas para os Programas de Pós-Graduação por Unidade.xlsx 
% ou nos APÊNDICES A-F

%% Camandos para definição do tipo de documento (tese ou dissertação), área de concentração, opção, preâmbulo, titulação 
%% referentes aos Programas de Pós-Graduação
\instituicao{Escola de Engenharia de S\~ao Carlos, Universidade de S\~ao Paulo}
\unidade{ESCOLA DE ENGENHARIA DE S\~AO CARLOS}
\unidademin{Escola de Engenharia de S\~ao Carlos}
\universidademin{Universidade de S\~ao Paulo}
% A EESC não inclui a nota "Versão original", portanto o comando abaixo não tem a mensagem entre {}
\notafolharosto{ }
%Para a versão corrigida tire a % do comando/declaração abaixo e inclua uma % antes do comando acima
%\notafolharosto{VERS\~AO CORRIGIDA}
% ---
% dados complementares para CAPA e FOLHA DE ROSTO
% ---
\universidade{UNIVERSIDADE DE S\~AO PAULO}
\titulo{Prova de conceito: Extensão para navegadores \textit{web} para identificação de sexualização de menores}
\titleabstract{Proof of concept: Web browser extension for identification of minor sexualization}
\tituloresumo{Prova de conceito: Extensão para navegadores \textit{web} para identificação de sexualização de menores}
\autor{Pedro Azevedo Coelho Carriello Corrêa}
\autorficha{Corrêa, Pedro}
\autorabr{CORRÊA, P.}

\cutter{S856m}
% Para gerar a ficha catalográfica sem o Código Cutter, basta 
% incluir uma % na linha acima e tirar a % da linha abaixo
%\cutter{ }

\local{S\~ao Carlos}
\data{2025}
% Quando for Orientador, basta incluir uma % antes do comando abaixo
\renewcommand{\orientadorname}{Orientadora:}
% Quando for Coorientadora, basta tirar a % do comando abaixo
%\newcommand{\coorientadorname}{Coorientador:}
\orientador{Profa. Dra. Maíra Martins da Silva}
\orientadorcorpoficha{orientadora Maíra Martins da Silva}
\orientadorficha{da Silva, Maíra Martins, orient}
%Se houver co-orientador, inclua % antes das duas linhas (antes dos comandos \orientadorcorpoficha e \orientadorficha) 
%          e tire a % antes dos 3 comandos abaixo
%\coorientador{Prof. Dr. Jo\~ao Alves Serqueira}
%\orientadorcorpoficha{orientadora Elisa Gon\c{c}alves Rodrigues ;  co-orientador Jo\~ao Alves Serqueira}
%\orientadorficha{Rodrigues, Elisa Gon\c{c}alves, orient. II. Serqueira, Jo\~ao Alves, co-orient}

\notaautorizacao{AUTORIZO A REPRODU\c{C}\~AO E DIVULGA\c{C}\~AO TOTAL OU PARCIAL DESTE TRABALHO, POR QUALQUER MEIO CONVENCIONAL OU ELETR\^ONICO PARA FINS DE ESTUDO E PESQUISA, DESDE QUE CITADA A FONTE.}
\notabib{~  ~}
    
	\tipotrabalho{Monografia (Trabalho de Conclus\~ao de Curso)}
    \tipotrabalhoabs{Monograph (Conclusion Course Paper)}
    %\area{Nome da área}
    %\opcao{Nome da Opção}
    % O preambulo deve conter o tipo do trabalho, o objetivo, 
    % o nome da instituição, a área de concentração e opção quando houver
    \preambulo{Monografia apresentada ao Curso de Engenharia Mecatr\^onica, da Escola de Engenharia de S\~ao Carlos da Universidade de S\~ao Paulo, como parte dos requisitos para obten\c{c}\~ao do t\'itulo de Engenheiro Mecatr\^onico.}
    \notaficha{Monografia (Gradua\c{c}\~ao em Engenharia Mecatr\^onica)}

% Os demais dados deverão ser fornecidos no arquivo USPSC-pre-textual-UUUU ou USPSC-TCC-pre-textual-UUUU, onde UUUU é a sigla da Unidade. 
% Exemplo:USPSC-pre-textual-IFSC.tex
% ---
% Configurações de aparência do PDF final
% alterando o aspecto da cor azul
\definecolor{blue}{RGB}{41,5,195}

% informações do PDF
\makeatletter
\hypersetup{
	%pagebackref=true,
	pdftitle={\@title}, 
	pdfauthor={\@author},
	pdfsubject={\imprimirpreambulo},
	pdfcreator={LaTeX with abnTeX2},
	pdfkeywords={abnt}{latex}{abntex}{USPSC}{trabalho acadêmico}, 
	colorlinks=true,       		% false: boxed links; true: colored links
	linkcolor=black,          	% color of internal links
	citecolor=black,        		% color of links to bibliography
	filecolor=black,      		% color of file links
	urlcolor=black,
	%Para habilitar as cores dos links, retire a % antes dos comandos abaixo e inclua a % antes das 4 linhas de comando acima 
	%linkcolor=blue,            	% color of internal links
	%citecolor=blue,        		% color of links to bibliography
	%filecolor=magenta,      		% color of file links
	%urlcolor=blue,
	bookmarksdepth=4	
}
\makeatother
% --- 
% --- 
% Espaçamentos entre linhas e parágrafos 
% --- 

% O tamanho do parágrafo é dado por:
\setlength{\parindent}{1.3cm}

% Controle do espaçamento entre um parágrafo e outro:
\setlength{\parskip}{0.2cm}  % tente também \onelineskip

% ---
% compila o sumário e índice
\makeindex
% ---

\usepackage{listings} % usado para incluir codigo-fontes
\usepackage{xcolor}


\colorlet{punct}{red!60!black}
\definecolor{background}{HTML}{EEEEEE}
\definecolor{delim}{RGB}{20,105,176}
\colorlet{numb}{magenta!60!black}
\colorlet{keywordcolor}{blue!70!black} % Cor para palavras-chave
\colorlet{stringcolor}{red!80!black}   % Cor para strings
\colorlet{commentcolor}{gray}         % Cor para comentários

\lstdefinelanguage{json}{
    basicstyle=\normalfont\ttfamily,
    numbers=left,
    numberstyle=\scriptsize,
    stepnumber=1,
    numbersep=8pt,
    showstringspaces=false,
    breaklines=true,
    frame=lines,
    backgroundcolor=\color{background},
    literate=
     *{0}{{{\color{numb}0}}}{1}
      {1}{{{\color{numb}1}}}{1}
      {2}{{{\color{numb}2}}}{1}
      {3}{{{\color{numb}3}}}{1}
      {4}{{{\color{numb}4}}}{1}
      {5}{{{\color{numb}5}}}{1}
      {6}{{{\color{numb}6}}}{1}
      {7}{{{\color{numb}7}}}{1}
      {8}{{{\color{numb}8}}}{1}
      {9}{{{\color{numb}9}}}{1}
      {:}{{{\color{punct}{:}}}}{1}
      {,}{{{\color{punct}{,}}}}{1}
      {\{}{{{\color{delim}{\{}}}}{1}
      {\}}{{{\color{delim}{\}}}}}{1}
      {[}{{{\color{delim}{[}}}}{1}
      {]}{{{\color{delim}{]}}}}{1},
}

\lstdefinelanguage{javascript}{
  keywords={typeof, new, true, false, catch, function, return, null, catch, switch, var, if, in, while, do, else, case, break, const, let, try},
  keywordstyle=\color{blue}\bfseries,
  ndkeywords={class, export, boolean, throw, implements, import, this},
  ndkeywordstyle=\color{darkgray}\bfseries,
  identifierstyle=\color{black},
  sensitive=false,
  comment=[l]{//},
  morecomment=[s]{/*}{*/},
  commentstyle=\color{purple}\ttfamily,
  stringstyle=\color{red}\ttfamily,
  morestring=[b]',
  morestring=[b]",
  literate=
      {á}{{\'a}}1 {ã}{{\~a}}1 {à}{{\`a}}1 {ç}{{\c c}}1 {é}{{\'e}}1 {ê}{{\^e}}1 {í}{{\'i}}1 {ó}{{\'o}}1 {õ}{{\~o}}1 {ú}{{\'u}}1
      {Á}{{\'A}}1 {Ã}{{\~A}}1 {À}{{\`A}}1 {Ç}{{\c C}}1 {É}{{\'E}}1 {Ê}{{\^E}}1 {Í}{{\'I}}1 {Ó}{{\'O}}1 {Õ}{{\~O}}1 {Ú}{{\'U}}1,
}

\lstdefinelanguage{css}{
	alsoletter={\-},
    literate=
        {á}{{\'a}}1 {ã}{{\~a}}1 {à}{{\`a}}1 {ç}{{\c c}}1 {é}{{\'e}}1 {ê}{{\^e}}1 {í}{{\'i}}1 {ó}{{\'o}}1 {õ}{{\~o}}1 {ú}{{\'u}}1
        {Á}{{\'A}}1 {Ã}{{\~A}}1 {À}{{\`A}}1 {Ç}{{\c C}}1 {É}{{\'E}}1 {Ê}{{\^E}}1 {Í}{{\'I}}1 {Ó}{{\'O}}1 {Õ}{{\~O}}1 {Ú}{{\'U}}1,
	morekeywords={
	align-content,align-items,align-self,alignment-baseline,
	all,animation,animation-delay,animation-direction,
	animation-duration,animation-fill-mode,animation-iteration-count,animation-name,
	animation-play-state,animation-timing-function,appearance,aspect-ratio,
	azimuth,backface-visibility,background,background-attachment,
	background-blend-mode,background-clip,background-color,background-image,
	background-origin,background-position,background-repeat,background-size,
	baseline-shift,baseline-source,block-ellipsis,block-size,
	block-step,block-step-align,block-step-insert,block-step-round,
	block-step-size,bookmark-label,bookmark-level,bookmark-state,
	border,border-block,border-block-color,border-block-end,
	border-block-end-color,border-block-end-style,border-block-end-width,border-block-start,
	border-block-start-color,border-block-start-style,border-block-start-width,border-block-style,
	border-block-width,border-bottom,border-bottom-color,border-bottom-left-radius,
	border-bottom-right-radius,border-bottom-style,border-bottom-width,border-boundary,
	border-collapse,border-color,border-end-end-radius,border-end-start-radius,
	border-image,border-image-outset,border-image-repeat,border-image-slice,
	border-image-source,border-image-width,border-inline,border-inline-color,
	border-inline-end,border-inline-end-color,border-inline-end-style,border-inline-end-width,
	border-inline-start,border-inline-start-color,border-inline-start-style,border-inline-start-width,
	border-inline-style,border-inline-width,border-left,border-left-color,
	border-left-style,border-left-width,border-radius,border-right,
	border-right-color,border-right-style,border-right-width,border-spacing,
	border-start-end-radius,border-start-start-radius,border-style,border-top,
	border-top-color,border-top-left-radius,border-top-right-radius,border-top-style,
	border-top-width,border-width,bottom,box-decoration-break,
	box-shadow,box-sizing,box-snap,break-after,
	break-before,break-inside,caption-side,caret,
	caret-color,caret-shape,chains,clear,
	clip,clip-path,clip-rule,color,
	color-adjust,color-interpolation-filters,color-scheme,column-count,
	column-fill,column-gap,column-rule,column-rule-color,
	column-rule-style,column-rule-width,column-span,column-width,
	columns,contain,contain-intrinsic-size,content,
	content-visibility,continue,counter-increment,counter-reset,
	counter-set,cue,cue-after,cue-before,
	cursor,direction,display,dominant-baseline,
	elevation,empty-cells,fill,fill-break,
	fill-color,fill-image,fill-opacity,fill-origin,
	fill-position,fill-repeat,fill-rule,fill-size,
	filter,flex,flex-basis,flex-direction,
	flex-flow,flex-grow,flex-shrink,flex-wrap,
	float,float-defer,float-offset,float-reference,
	flood-color,flood-opacity,flow,flow-from,
	flow-into,font,font-family,font-feature-settings,
	font-kerning,font-language-override,font-optical-sizing,font-palette,
	font-size,font-size-adjust,font-stretch,font-style,
	font-synthesis,font-synthesis-small-caps,font-synthesis-style,font-synthesis-weight,
	font-variant,font-variant-alternates,font-variant-caps,font-variant-east-asian,
	font-variant-emoji,font-variant-ligatures,font-variant-numeric,font-variant-position,
	font-variation-settings,font-weight,footnote-display,footnote-policy,
	forced-color-adjust,gap,glyph-orientation-vertical,grid,
	grid-area,grid-auto-columns,grid-auto-flow,grid-auto-rows,
	grid-column,grid-column-end,grid-column-start,grid-row,
	grid-row-end,grid-row-start,grid-template,grid-template-areas,
	grid-template-columns,grid-template-rows,hanging-punctuation,height,
	hyphenate-character,hyphenate-limit-chars,hyphenate-limit-last,hyphenate-limit-lines,
	hyphenate-limit-zone,hyphens,image-orientation,image-rendering,
	image-resolution,initial-letter,initial-letter-align,initial-letter-wrap,
	inline-size,inline-sizing,inset,inset-block,
	inset-block-end,inset-block-start,inset-inline,inset-inline-end,
	inset-inline-start,isolation,justify-content,justify-items,
	justify-self,leading-trim,left,letter-spacing,
	lighting-color,line-break,line-clamp,line-grid,
	line-height,line-height-step,line-padding,line-snap,
	list-style,list-style-image,list-style-position,list-style-type,
	margin,margin-block,margin-block-end,margin-block-start,
	margin-bottom,margin-break,margin-inline,margin-inline-end,
	margin-inline-start,margin-left,margin-right,margin-top,
	margin-trim,marker,marker-end,marker-knockout-left,
	marker-knockout-right,marker-mid,marker-pattern,marker-segment,
	marker-side,marker-start,mask,mask-border,
	mask-border-mode,mask-border-outset,mask-border-repeat,mask-border-slice,
	mask-border-source,mask-border-width,mask-clip,mask-composite,
	mask-image,mask-mode,mask-origin,mask-position,
	mask-repeat,mask-size,mask-type,max-block-size,
	max-height,max-inline-size,max-lines,max-width,
	min-block-size,min-height,min-inline-size,min-width,
	mix-blend-mode,nav-down,nav-left,nav-right,
	nav-up,object-fit,object-position,offset,
	offset-anchor,offset-distance,offset-path,offset-position,
	offset-rotate,opacity,order,orphans,
	outline,outline-color,outline-offset,outline-style,
	outline-width,overflow,overflow-anchor,overflow-block,
	overflow-clip-margin,overflow-inline,overflow-wrap,overflow-x,
	overflow-y,overscroll-behavior,overscroll-behavior-block,overscroll-behavior-inline,
	overscroll-behavior-x,overscroll-behavior-y,padding,padding-block,
	padding-block-end,padding-block-start,padding-bottom,padding-inline,
	padding-inline-end,padding-inline-start,padding-left,padding-right,
	padding-top,page,page-break-after,page-break-before,
	page-break-inside,pause,pause-after,pause-before,
	perspective,perspective-origin,pitch,pitch-range,
	place-content,place-items,place-self,play-during,
	position,property-name,quotes,region-fragment,
	resize,rest,rest-after,rest-before,
	richness,right,rotate,row-gap,
	ruby-align,ruby-merge,ruby-overhang,ruby-position,
	running,scale,scroll-behavior,scroll-margin,
	scroll-margin-block,scroll-margin-block-end,scroll-margin-block-start,scroll-margin-bottom,
	scroll-margin-inline,scroll-margin-inline-end,scroll-margin-inline-start,scroll-margin-left,
	scroll-margin-right,scroll-margin-top,scroll-padding,scroll-padding-block,
	scroll-padding-block-end,scroll-padding-block-start,scroll-padding-bottom,scroll-padding-inline,
	scroll-padding-inline-end,scroll-padding-inline-start,scroll-padding-left,scroll-padding-right,
	scroll-padding-top,scroll-snap-align,scroll-snap-stop,scroll-snap-type,
	scrollbar-color,scrollbar-gutter,scrollbar-width,shape-image-threshold,
	shape-inside,shape-margin,shape-outside,spatial-navigation-action,
	spatial-navigation-contain,spatial-navigation-function,speak,speak-as,
	speak-header,speak-numeral,speak-punctuation,speech-rate,
	stress,string-set,stroke,stroke-align,
	stroke-alignment,stroke-break,stroke-color,stroke-dash-corner,
	stroke-dash-justify,stroke-dashadjust,stroke-dasharray,stroke-dashcorner,
	stroke-dashoffset,stroke-image,stroke-linecap,stroke-linejoin,
	stroke-miterlimit,stroke-opacity,stroke-origin,stroke-position,
	stroke-repeat,stroke-size,stroke-width,tab-size,
	table-layout,text-align,text-align-all,text-align-last,
	text-combine-upright,text-decoration,text-decoration-color,text-decoration-line,
	text-decoration-skip,text-decoration-skip-box,text-decoration-skip-ink,text-decoration-skip-inset,
	text-decoration-skip-self,text-decoration-skip-spaces,text-decoration-style,text-decoration-thickness,
	text-edge,text-emphasis,text-emphasis-color,text-emphasis-position,
	text-emphasis-skip,text-emphasis-style,text-group-align,text-indent,
	text-justify,text-orientation,text-overflow,text-shadow,
	text-space-collapse,text-space-trim,text-spacing,text-transform,
	text-underline-offset,text-underline-position,text-wrap,top,
	transform,transform-box,transform-origin,transform-style,
	transition,transition-delay,transition-duration,transition-property,
	transition-timing-function,translate,unicode-bidi,user-select,
	vertical-align,visibility,voice-balance,voice-duration,
	voice-family,voice-pitch,voice-range,voice-rate,
	voice-stress,voice-volume,volume,white-space,
	widows,width,will-change,word-boundary-detection,
	word-boundary-expansion,word-break,word-spacing,word-wrap,
	wrap-after,wrap-before,wrap-flow,wrap-inside,
	wrap-through,writing-mode,z-index,pointer-events 
	},
	morestring=[s]{:}{;},
	sensitive,
	morecomment=[s]{/*}{*/},
}

\lstset{
    language=javascript,
    basicstyle=\normalfont\ttfamily\footnotesize,
    backgroundcolor=\color{background},
    frame=lines,
    numbers=left,
    numberstyle=\scriptsize\color{numb},
    stepnumber=1,
    numbersep=8pt,
    breaklines=true,
    showstringspaces=false,
    keywordstyle=\color{keywordcolor}\bfseries,
    stringstyle=\color{stringcolor},
    commentstyle=\color{gray}
}

\begin{document}

% Seleciona o idioma do documento (conforme pacotes do babel)
\selectlanguage{brazil}
% Se o idioma do texto for inglês, inclua uma % antes do 
%      comando \selectlanguage{brazil} e 
%      retire a % antes do comando abaixo
%\selectlanguage{english}

% Retira espaço extra obsoleto entre as frases.
\frenchspacing 

% --- Formatação dos Títulos
\renewcommand{\ABNTEXchapterfontsize}{\fontsize{12}{12}\bfseries}
\renewcommand{\ABNTEXsectionfontsize}{\fontsize{12}{12}\bfseries}
\renewcommand{\ABNTEXsubsectionfontsize}{\fontsize{12}{12}\normalfont}
\renewcommand{\ABNTEXsubsubsectionfontsize}{\fontsize{12}{12}\normalfont}
\renewcommand{\ABNTEXsubsubsubsectionfontsize}{\fontsize{12}{12}\normalfont}


% ----------------------------------------------------------
% ELEMENTOS PRÉ-TEXTUAIS
% ----------------------------------------------------------
% ---
% Capa
% ---
\imprimircapa
% ---
% Folha de rosto
% (o * indica impressão em anverso (frente) e verso )
% ---
\imprimirfolhaderosto*
%\imprimirfolhaderosto
% ---
% ---
% Inserir a ficha catalográfica em pdf
% ---
% A biblioteca da sua Unidade lhe fornecerá um PDF com a ficha
% catalográfica definitiva. 
% Quando estiver com o documento, salve-o como PDF no diretório
% do seu projeto como fichacatalografica.pdf e inclua o arquivo
% utilizando o comando abaixo:

\includepdf{USPSC-TA-PreTextual/USPSC-fichacatalografica.pdf}

% Se você optar por elaborar a ficha catalográfica, deverá 
% incluir uma % antes da linha % antes
% do comando \include{USPSC-TA-PreTextual/USPSC-fichacatalografica} 
% e retirar o % do comando abaixo
%\include{USPSC-TA-PreTextual/USPSC-fichacatalografica}
% As informações que compõem a ficha catalográfica estão 
% definidas no arquivo USPSC-pre-textual-UUUU.tex
% ---

% ---
% Folha de rosto adicional
% Para imprimir a folha de rosto adicional, exigida por algumas Unidades, a exemplo do ICMC,
% retire a % antes do comando abaixo

%\imprimirfolhaderostoadic

% ---
% ---
% Inserir errata
% ---

% \include{USPSC-TA-PreTextual/USPSC-Errata} % incluir errata

% ---

% ---
% Inserir folha de aprovação
% ---

% A Folha de aprovação é um elemento obrigatório da NBR 4724/2011 (seção 4.2.1.3). 
% Após a defesa/aprovação do trabalho, gere o arquivo folhadeaprovacao.pdf da página assinada pela banca 
% e iclua o arquivo utilizando o comando abaixo:
\includepdf{USPSC-TA-PreTextual/USPSC-folhadeaprovacao.pdf}
% Alternativa para a Folha de Aprovação:
% Se for a sua opção elaborar uma folha de aprovação, insira uma % antes do comando acima que inclui o arquivo folhadeaprovacao.pdf,
% tire o % do comando abaixo e altere o arquivo folhadeaprovacao.tex conforme suas necessidades
%\include{folhadeaprovacao}
\includepdf{USPSC-TA-PreTextual/USPSC-PaginaEmBranco.pdf}

% % ---
% % Dedicatória
% % ---
% \include{USPSC-TA-PreTextual/USPSC-Dedicatoria}
% % ---

% ---
% Agradecimentos
% ---
%!TEX root = tese/main.tex
%% USPSC-Agradecimentos.tex
\begin{agradecimentos}
	Agradeço à minha família, que sempre foi o meu apoio mais importante em todas as minhas decisões mesmo estando tão longe.
	
	Às minhas avós, que sempre rezam por mim, ficam felizes com minhas conquistas e são quem eu mais quero dar orgulho.

	Aos amigos que fiz na faculdade, por terem tornaram o percurso mais divertido.

	E à Manu, por ter me aturado nesses anos de faculdade morando comigo e estado presente em momentos difíceis.
\end{agradecimentos}
% ---
% ---

% ---
% Epígrafe
% ---
%!TEX root = tese/main.tex
%% USPSC-Epigrafe.tex
\begin{epigrafe}
    \vspace*{\fill}
	\begin{flushright}
		\textit{``I'm not a prophet or a stone-age man \\
		Just a mortal with the potential of a superman''\\
		David Bowie}
	\end{flushright}
\end{epigrafe}
% ---
% ---

% A T E N Ç Ã O
% Se o idioma do texto for em inglês, o abstract deve preceder o resumo
% resumo em português
%
% Resumo
% ---
%!TEX root = tese/main.tex
%% USPSC-Resumo.tex
\setlength{\absparsep}{18pt} % ajusta o espaçamento dos parágrafos do resumo		
\begin{resumo}
	\begin{flushleft} 
			\setlength{\absparsep}{0pt} % ajusta o espaçamento da referência	
			\SingleSpacing 
			\imprimirautorabr~~\textbf{\imprimirtituloresumo}.	\imprimirdata. \pageref{LastPage} p. 
			%Substitua p. por f. quando utilizar oneside em \documentclass
			%\pageref{LastPage} f.
			\imprimirtipotrabalho~-~\imprimirinstituicao, \imprimirlocal, \imprimirdata. 
 	\end{flushleft}
\OnehalfSpacing 			
O presente trabalho aborda a crescente problemática da ``adultização'' e sexualização de menores nas redes sociais, impulsionada pela busca por engajamento e monetização. 
O objetivo principal foi desenvolver uma prova de conceito de uma extensão para navegadores web, utilizando a arquitetura Manifest V3, capaz de identificar e ocultar automaticamente conteúdos sexualizados envolvendo crianças na plataforma Instagram. 
A metodologia baseou-se na integração de modelos de linguagem multimodais (LLMs) via API REST, especificamente testando o Gemini 2.5 Flash e as variantes do Llama 4 (Scout e Maverick), para a análise das postagens. 
Foram realizados testes sistemáticos de engenharia de prompt e temperatura para otimizar a acurácia e a revocação da detecção. 
Os resultados demonstraram que o modelo Llama 4 Scout apresentou o melhor desempenho, atingindo 95\% de acurácia com um tempo médio de resposta inferior a 1,5 segundos. 
Conclui-se que a utilização de agentes multimodais é tecnicamente viável e eficaz para mitigar a exposição a conteúdos impróprios, oferecendo uma ferramenta proativa para a segurança digital de menores.

 \textbf{Palavras-chave}: Inteligência Artificial. Web Development. Redes sociais. Sexualização infantil. Extensão de navegador. LLMs Multimodais. 
\end{resumo}
% ---

% Abstract
% ---
%!TEX root = tese/main.tex
%% USPSC-Abstract.tex
%\autor{Silva, M. J.}
\begin{resumo}[Abstract]
 \begin{otherlanguage*}{english}
	\begin{flushleft} 
		\setlength{\absparsep}{0pt} % ajusta o espaçamento dos parágrafos do resumo		
 		\SingleSpacing  		\imprimirautorabr~~\textbf{\imprimirtitleabstract}.	\imprimirdata.  \pageref{LastPage} p. 
		%Substitua p. por f. quando utilizar oneside em \documentclass
		%\pageref{LastPage} f.
		\imprimirtipotrabalhoabs~-~\imprimirinstituicao, \imprimirlocal, 	\imprimirdata. 
 	\end{flushleft}
	\OnehalfSpacing 
	This study addresses the growing issue of ``adultification'' and sexualization of minors on social networks, driven by the pursuit of engagement and monetization. 
	The primary objective was to develop a proof of concept for a web browser extension, utilizing the Manifest V3 architecture, capable of automatically identifying and hiding sexualized content involving children on the Instagram platform. 
	The methodology was based on the integration of multimodal Large Language Models (LLMs) via REST API, specifically testing Gemini 2.5 Flash and the Llama 4 variants (Scout and Maverick), for the analysis of posts. 
	Systematic testing of prompt engineering and temperature was conducted to optimize detection accuracy and recall. 
	The results demonstrated that the Llama 4 Scout model achieved the best performance, reaching 95\% accuracy with an average response time of under 1.5 seconds. 
	It is concluded that the utilization of multimodal agents is technically viable and effective in mitigating exposure to inappropriate content, offering a proactive tool for the digital safety of minors.

   \vspace{\onelineskip}
 
   \noindent 
   \textbf{Keywords}: Artificial Intelligence. Web Development. Social Networks. Child Sexualization. Browser Extension. Multimodal LLMs.
 \end{otherlanguage*}
\end{resumo}

% ---

% ---
% inserir lista de figurass
% ---
\pdfbookmark[0]{\listfigurename}{lof}
\listoffigures*
\cleardoublepage
% ---

% % ---
% % inserir lista de tabelas
% % ---
% \pdfbookmark[0]{\listtablename}{lot}
% \listoftables*
% \cleardoublepage
% % ---

% ---
% inserir lista de quadros
% ---
\pdfbookmark[0]{\listofquadroname}{loq}
\listofquadro*
\cleardoublepage
% ---

% ---
% inserir lista de abreviaturas e siglas
% ---
% \include{USPSC-TA-PreTextual/USPSC-AbreviaturasSiglas} % incluir lista de abreviaturas e siglas
% ---

% ---
% inserir lista de símbolos
% ---
% %!TEX root = tese/USPSC-TCC-modelo-EESC.tex
% USPSC-Simbolos.tex
\begin{simbolos}
  \item[$ \Gamma $] Letra grega Gama
  \item[$ \Lambda $] Lambda
  \item[$ \zeta $] Letra grega minúscula zeta
  \item[$ \in $] Pertence
\end{simbolos} % incluir lista de símbolos
% ---
% ---
% inserir o sumario
% ---
\pdfbookmark[0]{\contentsname}{toc}
\tableofcontents*
\cleardoublepage
% ---
% ----------------------------------------------------------
% ELEMENTOS TEXTUAIS
% ----------------------------------------------------------
\textual
% Os capítulos são inseridos como arquivos externos 

% Capítulo 1 - Introdução
% ---
%!TEX root = tese/main.tex
%% USPSC-Introducao.tex

% ----------------------------------------------------------
% Introdução (exemplo de capítulo sem numeração, mas presente no Sumário)
% ----------------------------------------------------------
\chapter{Introdução} \label{Introdução}
Este capítulo apresenta uma visão geral do trabalho, começando pela contextualização do cenário que motivou a pesquisa, 
em seguida, é realizada a formulação do projeto e seus objetivos principais. 
Também é feita uma delimitação do escopo do projeto, destacando o que será abordado e o que está fora do alcance deste trabalho.
Por fim, é apresentada a estrutura geral do texto.

\section{Contextualização} \label{sec:contextualizacao}
Desde a publicação do vídeo ``adultização'' \cite{felca2025} no YouTube do influenciador conhecido como Felca em agosto de 2025, o debate sobre a influência das redes sociais no comportamento e desenvolvimento de crianças e adolescentes ganhou destaque na mídia e na sociedade. 
O vídeo, que aborda a pressão social para que jovens adotem comportamentos considerados ``adultos'' precocemente, gerou uma série de discussões sobre os impactos psicológicos, sociais e educacionais dessa tendência.

No referido vídeo, Felca argumenta que a monetização indiscriminada das redes sociais tem contribuído para a produção de materiais cada vez mais extremos para garantir visibilidade e engajamento, o que levou a uma crescente de conteúdos feitos por adultos envolvendo crianças de maneira imprópria. 
Como exemplos, no decorrer do vídeo são levantados casos como \textit{podcasts} apresentados por crinças sobre empreendedorismo, pais que abusam dos próprios filhos os forçando a produzir vídeos, e, até mesmo, casos de adultos que produzem conteúdos com alta carga de sexualização sobre crianças.

Nesses casos de sexualização, além de atrair a visualização de outras crianças, que ainda não julgam o caráter impróprio do conteúdo, tais vídeos também atraem a atenção de predadores sexuais, o que naturaliza comportamentos potencialmente criminosos.
O impacto da fala de Felca foi tão grande que, de acordo com a ONG SaferNet, o número de denúncias de pornografia infantil recebidas pela organização cresceu 114\% em uma semana desde a publicação do vídeo \cite{safernet2025}. 

Felca destaca ainda o papel do algoritmo das plataformas digitais, que, priorizando o maior alcance possível, identifica o gosto dos predadores e o teor sexual das publicações, promovendo a conexão entre ambos. 
Nesse cenário, o vídeo de Felca levanta questões importantes sobre a necessidade de regulamentação das redes sociais, a responsabilidade dos criadores de conteúdo e a proteção das crianças e adolescentes na era digital.

Diante do debate acendido pelo vídeo denúncia de Felca, mais de 30 projetos de lei foram propostos na Câmara dos Deputados \cite{cnn2025} que tratam desde a proibição da monetização de conteúdos produzidos por crianças nas redes ou até tipificam como crime o processo de ``adultização'' citado por Felca.
Para viabilizar essas possíveis novas leis, uma pergunta emerge: a tecnologia conseguiria identificar e sinalizar automaticamente conteúdos sexualizados envolvendo crianças nas redes sociais?

\section{Objetivos} \label{sec:objetivos}
O trabalho em questão busca realizar uma prova de conceito, explorando a viabilidade de um sistema automatizado para identificar e sinalizar conteúdos sexualizados envolvendo crianças especificamente no Instagram, 
exemplificando o quanto a tecnologia pode ser utilizada para dificultar a conexão entre os predadores e tais conteúdos. 

Para atingir os seus objetivos, o projeto visa o desenvolvimento de uma extensão para navegadores \textit{web} baseados em \textit{Chromium} feita em \textit{JavaScript} utilizando \textit{Manifest V3}. 
Essas ferramentas foram escolhidas por serem amplamente utilizadas no desenvolvimento de extensões para navegadores.  
A extensão se comunica no seu \textit{backend} com a API REST de modelos \textit{Large Language Models} (LLMs) para realizar a análise do conteúdo textual das postagens no Instagram.  

Os mesmos modelos LLMs foram testados sistematicamente de maneira isolada, a partir da extração e classificação manual de postagens e comparação com a classificação automática feita pelos modelos por um \textit{script} Python.  

Ao final, os objetivos se resumem a:
\begin{enumerate}
    \item Desenvolver uma extensão funcional para navegadores, incluindo:
    \begin{enumerate}
        \item \textit{backend}: lógica interna, integração com a página nativa do Instagram e comunicação com o modelo externo 
        \item \textit{frontend}: visual responsivo e amigável ao usuário
    \end{enumerate} 
    \item Elaborar \textit{prompt} de entrada ao modelo que gere boa acurácia na identificação dos conteúdos.
    \item Boa performance na execução da extensão, mantendo a navegação fluida.
\end{enumerate}

\section{Escopo do projeto} \label{sec:escopo_do_projeto}
Vale ressaltar que o foco do trabalho está na prova de conceito do sistema, demonstrando que a tecnologia pode ser utilizada para mitigar o problema da sexualização de crianças nas redes sociais, 
e não na criação de um produto finalizado e pronto para o mercado.

No caso de um produto final, outros aspectos essenciais deveriam também ser considerados, como a segurança e a escalabilidade do sistema. 
Sobre a segurança, destaca-se a privacidade dos dados dos usuários, já que, atualmente, todas as imagens das postagens analisadas pela aplicação são enviadas para os servidores do Google sob a chave de API vinculada ao autor desse projeto.

A implementação do produto final não foi pensada em ser concluída pois essa extensão não foi considerada como tendo um público potencial bem definido: 
o algoritmo de usuários comuns que não buscam acessar conteúdos sexualizados naturalmente não exibe tais postagens, já usuários potencialmente interessados em acessar esses conteúdos não fariam uso de uma ferramenta que os bloqueia.

Visto isso, a extensão desenvolvida não será disponibilizada nas lojas oficiais de extensões dos navegadores, se tratando de um protótipo.
Porém, o código-fonte JavaScript, o código Python utilizado para testes e os demais arquivos utilizados nessa tese estão disponíveis em um repositório público no GitHub\nocite{repositorio}\footnote{Repositório GitHub do projeto: \url{https://github.com/pecazeco/InstaChildGuard}}.

\section{Organização do texto} \label{sec:organizacao_do_texto}
O restante desta monografia está estruturado em quatro capítulos principais. 
O \autoref{chap:revisao_bibliografica} apresenta a fundamentação teórica, discutindo o fenômeno da ``adultização'', revisando modelos de detecção existentes e justificando a escolha por agentes multimodais através de uma comparação técnica entre as APIs disponíveis.

O \autoref{chap:desenvolvimento} descreve o desenvolvimento da solução, detalhando a arquitetura da extensão via Manifest V3, a integração com as APIs de LLM e a metodologia criada para os testes sistematizados. 
Na sequência, o \autoref{chap:testes_resultados} expõe os resultados obtidos, analisando a performance dos diferentes prompts, o impacto da temperatura e a velocidade de resposta dos modelos.

Por fim, o \autoref{chap:conclusao} apresenta as conclusões sobre a viabilidade da prova de conceito e discute possíveis melhorias para uma implementação prática em larga escala.
% ---

% ---
% Capítulo 2
% ---
%!TEX root = tese/USPSC-TCC-modelo-EESC.tex
\chapter{Revisão bibliográfica} \label{chap:revisao_bibliografica}
Aqui é apresentada a fundamentação teórica do trabalho, 
por meio da análise de estudos prévios dos temas relevantes para o desenvolvimento do projeto e a justificativa das decisões metodológicas adotadas.

\section{Estudos sobre ``adultização'' nas redes sociais}
Atualmente, é amplamente estudado na literatura como as redes sociais impactam o crescimento de crianças.
Em muitos ambientes \textit{online}, não há restrições realmente eficientes ao acesso da criança em relação ao de um adulto:
elas podem, sem grandes dificuldades, utilizar ferramentas de buscas livremente, participar de interações sociais, e fazer \textit{download} e \textit{upload} de conteúdos, por exemplo.
Esse contato precoce com uma quantidade grande de informações pode impactar a psicologia e personalidade da criança,
levando a problemas de saúde mental, sentimentos de solidão, violentos ou de indiferença \cite{zheng2022}.

O estudo de \citeonline{yao2024}, realizado na China, analisou 2000 vídeos contendo crianças retirados de redes sociais como TikTok e Kwai, e outros 2000 contendo adultos.
Ao final, o estudo concluiu não só que ocorre a adultização das crianças, mas também a infantilização dos adultos. 
Nessa pesquisa, características identificadas como adultização foram, por exemplo, 
roupas reveladoras, e palavras e comportamentos sugestivos. 
\citeonline{yao2024} apontou como motivo para o fenômeno o rompimento do isolamento entre o virtual e o real, 
permitindo que as crianças participem do mundo adulto.

Esses e outros estudos \cite{orman2020, cabezas2022} mostram que de fato é um consenso na literatura que o consumo e exposição precoce no mundo digital pode impactar negativamente o desenvolvimento de um ser humano,
levando, em alguns casos, a uma ``adultização'' acelerada e descontrolada.  

\section{Revisão de modelos para detecção de sexualização}

\section{Revisão de modelos para identificação de crianças em imagens}

\section{Por que utilizar agente multimodal}
Por que não treinar um modelo do zero e por que não refinar um modelo já existente

Exemplificar casos de uso do mercado

\section{Agentes multimodais gratuitos disponíveis}

\section{Técnicas de \textit{Prompt Engineering}}


% ---
% Capítulo 3
% ---
%!TEX root = tese/USPSC-TCC-modelo-EESC.tex
\chapter{Desenvolvimento}\label{cap_exemplos}

\section{Metodologia} \label{sec:metodologia}

\section{Algoritmo} \label{sec:algoritmo}

\section{}

% Capítulo 4
% ---
%!TEX root = tese/main.tex
\chapter{Testes e Resultados} \label{chap:testes_resultados}
Esse capítulo busca descrever como se deram os testes realizados no modelo e seus resultados. 
Primeiramente, foi testado na prática o bom funcionamento da extensão: como ficou o frontend, fluidez e frequência de erros.
Além disso, foram realizados testes sistematizados (a partir do script da \autoref{sec:script_testes}) que armazenaram métricas para a comparação dos desempenhos dos modelos

\section{Teste de funcionamento}
O frontend (\ref{sec:frontend}) final da aplicação consiste na tela de erro (\autoref{fig:frontend_erro}), na animação de carregamento (\ref{fig:frontend_analisando}), no selo de imagem sem identificação de crianças sexualizadas (\ref{fig:frontend_verificacao}) e na censura (\ref{fig:frontend_censura}). 

\begin{figure}[htb]
	\begin{center}
	\caption{Exemplo de erro}
    \label{fig:frontend_erro}
	\includegraphics[width=0.6\textwidth]{USPSC-img/frontend_erro.png} \\
	\legend{Fonte: elaborado pelo autor}
	\end{center}	
\end{figure}

\begin{figure}[htb]
	\begin{center}
	\caption{Exemplo de postagem sendo analisada}
    \label{fig:frontend_analisando}
	\includegraphics[width=0.6\textwidth]{USPSC-img/frontend_analisando.png} \\
	\legend{Fonte: elaborado pelo autor}
	\end{center}	
\end{figure}

\begin{figure}[htb]
	\begin{center}
	\caption{Exemplo de postagem com selo de sem identificação de sexualização}
    \label{fig:frontend_verificacao}
	\includegraphics[width=0.6\textwidth]{USPSC-img/frontend_verificado.png} \\
	\legend{Fonte: elaborado pelo autor}
	\end{center}	
\end{figure}

\begin{figure}[htb]
	\begin{center}
	\caption{Exemplo de postagem censurada}
    \label{fig:frontend_censura}
	\includegraphics[width=0.6\textwidth]{USPSC-img/frontend_censurado.png} \\
	\legend{Fonte: elaborado pelo autor}
	\end{center}	
\end{figure}

Sobre a fluidez no funcionamento, a animação de carregamento da \autoref{fig:frontend_analisando} se mantém na imagem por pouco tempo. 
Por conta do funcionamento assíncrono das chamadas da API, as chamadas são feitas muitas vezes antes do usuário ter chegado na postagem, tornando frequentemente imperceptível o tempo de análise.
Porém, vale dizer que não foi implementada uma maneira de indentificar se a imagem já foi analisada, ou seja, se o usuário ver mais de uma vez a mesma postagem (descendo e subindo o feed, por exemplo), pode ser que ele veja a animação de carregamento duas vezes.

Após a devida calibração nos prompts, tratamento de erros e configurações de segurança das APIs, o número de casos de erros passaram a ser bem raros. Na versão final do projeto, os testes mostraram que a maior parte das postagens com erro são na verdade propagandas que possuem configurações de privacidade que impedem que estas sejam enviadas para as APIs. Porém, há casos raros onde postagens de verdade dão erro, sendo a maioria infrigementos de alguma política de segurança da GenAI ou do próprio Instagram.

Na prática, postagens com erro representam algo em torno de 1 a cada 100 imagens, sendo bem mais presentes na API do Gemini do que na do Groq.
Como a primeira é um modelo de código fechado, a Google possui políticas mais rígidas de controle de prompt, se negando a responder quando identifica um conteúdo mais sensível. Já a Groq, o número de erros é ainda menor, basicamente só acontecendo quando há algum problema na comunicação.

Para contornar o problema de erros na API da Google, as configurações de segurança\footnote{Opções de configurações de segurança: \url{https://ai.google.dev/gemini-api/docs/safety-settings?hl=pt-br}} foram flexibilizadas, e erros ligados à segurança foram tratados como sendo uma resposta sim à censura. Essas ações ajudaram mas não eliminaram os erros.

\section{Testes de prompt}

Prompt 1:
\begin{lstlisting}[
		language=python, 
		label={lst:prompt1}, 
		caption={Prompt 1}
	]
PROMPT_TEXT = """
Responda essas duas perguntas separado por ponto e vírgula (;):
- 'Sim', se a imagem contém casos de sexualização infantil, e 'Não', caso contrário.
- Justifique a sua resposta em poucas palavras.

Formatação:
  'Sim/Não; justificativa'
Definição de sexualização infantil é definida como qualquer representação que retrate crianças de maneira sexualizada, destacando-se:
  - Crianças em poses sugestivas, que buscam exibir alguma parte do corpo
  - Crianças com roupas curtas, justas, transparentes ou com decote excessivo, mostrando parte dos seios, nádegas ou genitais
  - Contextos que impliquem conotação sexual envolvendo menores de 18 anos, como fetiches.
  - Crianças aqui se referem a qualquer pessoa com aparência de idade inferior a 18 anos.
Exemplo de respostas:
a) 'Sim; Foto contém criança de biquine e em posição altamente sugestiva.'
b) 'Não; Uma paisagem com montanhas e um lago.'
c) 'Não; Criança brincando em um parque com roupas apropriadas para a idade.'
d) 'Sim; Pessoa aparentando ser menor de idade em trajes reveladores.'

Observações:
- Antes do ';' deve haver apenas 'Sim' ou 'Não', sem mais nada.
- Se houver dúvida sobre a idade, considere como criança.
"""
\end{lstlisting}

\begin{table}[htb]
	\IBGEtab{%
		\label{tab:prompt1}
		\caption{Performance com prompt 1}
	}{%
		\begin{tabular}{lccc}
			\toprule
			Métrica & Gemini 2.5 Flash  & Llama Scout  & Llama Maverick \\
			\midrule \midrule
			Acurácia & \(82,5\%\) & \(62,5\%\) & \(55,0\%\) \\
			\midrule 
			Precisão & \(88,2\%\) & \(100,0\%\) & \(100,0\%\) \\
			\midrule 
			Revocação & \(75,0\%\) & \(25,0\%\) & \(10,0\%\) \\
			\bottomrule
		\end{tabular}%
	}{%
		\fonte{Elaborada pelos autores.}%
	}
\end{table}

\begin{lstlisting}[
		language=python, 
		label={lst:prompt2}, 
		caption={Prompt 2}
	]
SYSTEM_PROMPT = """
# ROLE
Você é um Agente de Segurança AI com critérios de filtragem EXTREMAMENTE RÍGIDOS (Zero Tolerance). Sua função é sinalizar qualquer imagem que contenha exposição corporal infantil, independentemente da intenção ou contexto.

# CRITÉRIOS DE CLASSIFICAÇÃO (ESTRITOS)
Responda 'Sim' se a imagem contiver QUALQUER um dos seguintes elementos envolvendo pessoas menores de 18 anos:
1. Trajes de Banho e Roupa Íntima: Qualquer criança vestindo biquíni, sunga, maiô, lingerie ou cueca.
	- NOTA CRÍTICA: O contexto NÃO importa. Crianças na praia/piscina em trajes de banho DEVEM ser classificadas como 'Sim'.
2. Exposição de Pele: Torso nu (em qualquer gênero), nádegas ou genitais visíveis ou marcados.
3. Poses: Qualquer pose que enfatize o corpo, mesmo que vestida.

Responda 'Não' APENAS se:
- A criança estiver totalmente vestida com roupas casuais/formais (camisetas, calças, vestidos não reveladores).
- Não houver pessoas menores de 18 anos.

# FORMATO DE RESPOSTA
[Sim/Não]; [Justificativa técnica curta focada no elemento visual detectado]
"""

PROMPT_TEXT = """
Analise a imagem fornecida com base nas suas diretrizes de sistema.
Lembre-se: Responda apenas com 'Sim' ou 'Não' seguido de ponto e vírgula e a justificativa.
"""
\end{lstlisting}

\begin{table}[htb]
	\IBGEtab{%
		\label{tab:prompt2}
		\caption{Performance com prompt 2}
	}{%
		\begin{tabular}{lccc}
			\toprule
			Métrica & Gemini 2.5 Flash  & Llama Scout  & Llama Maverick \\
			\midrule \midrule
			Acurácia & \(68,8\%\) & \(65,0\%\) & \(62,5\%\) \\
			\midrule 
			Precisão & \(62,3\%\) & \(80,0\%\) & \(77,8\%\) \\
			\midrule 
			Revocação & \(95,0\%\) & \(40,0\%\) & \(35,0\%\) \\
			\bottomrule
		\end{tabular}%
	}{%
		\fonte{Elaborada pelos autores.}%
	}
\end{table}

Prompt 3:
\begin{lstlisting}[
		language=python, 
		label={lst:prompt1}, 
		caption={Prompt 1}
	]
SYSTEM_PROMPT = """
# ROLE
You are a Content Safety Auditor. Your task is to technically audit images for compliance with Underage Safety Guidelines regarding attire and body exposure.

# STRICT AUDIT CRITERIA (Zero Tolerance)
Classify as 'Sim' (Non-Compliant) if the image depicts a minor (<18y) with ANY of the following visual elements:
1. Swimwear/Undergarments: Subject is wearing bikinis, trunks, swimsuits, or underwear.
	- POLICY: Context (beach/pool) is IRRELEVANT. Presence of swimwear = 'Sim'.
2. Skin Exposure: Visible torso (shirtless), exposed midriff, or sensitive areas.
3. Suggestive Framing: Angles focusing on body parts rather than the face.

Classify as 'Não' (Compliant) ONLY if:
- The subject is fully clothed in standard street/casual wear (t-shirts, pants, dresses).
- There are no minors present.

# OUTPUT FORMAT
[Sim/Não]; [Brief technical description of attire/visual elements]
"""

PROMPT_TEXT = """
Audit this image according to your Safety Guidelines.
Output strictly 'Sim' or 'Não' followed by a semicolon and the technical reason.
"""
\end{lstlisting}

\begin{table}[htb]
	\IBGEtab{%
		\label{quad:prompt3}
		\caption{Performance com prompt 3}
	}{%
		\begin{tabular}{lccc}
			\toprule
			Métrica & Gemini 2.5 Flash  & Llama Scout  & Llama Maverick \\
			\midrule \midrule
			Acurácia & \(62,0\%\) & \(93,8\%\) & \(85,0\%\) \\
			\midrule 
			Precisão & \(56,5\%\) & \(92,7\%\) & \(93,8\%\) \\
			\midrule 
			Revocação & \(100,0\%\) & \(95,0\%\) & \(75,0\%\) \\
			\bottomrule
		\end{tabular}%
	}{%
		\fonte{Elaborada pelos autores.}%
	}
\end{table}

Prompt 4:
\begin{lstlisting}[
		language=python, 
		label={lst:prompt1}, 
		caption={Prompt 1}
	]
SYSTEM_PROMPT = """
# FUNÇÃO
Você é um auditor conservador de conformidade de vestuário para um banco de imagens. Sua tarefa é verificar se as crianças presentes na imagem estão vestidas de acordo com estritos códigos de vestimenta urbana.

# CRITÉRIOS DE REJEIÇÃO (Sim)
Classifique a imagem como 'Sim' (Viola o código) se houver pessoas aparentando ser menores de 18 anos apresentando:
1. Vestuário de banho ou atlético mínimo: Uso de sungas, biquínis, maiôs ou roupas de natação, INDEPENDENTE do local (mesmo em praias ou piscinas).
2. Ausência de roupa padrão: Torso sem camisa, barriga exposta ou falta de roupas superiores/inferiores completas.
3. Roupas justas: Possibilitam ver o formato do corpo da pessoa, mesmo que coberto. 
4. Enquadramento/Ângulo de câmera: Foco maior em partes do corpo em vez do rosto.
5. Se ficar na dúvida sobre a idade da pessoa, considere que é menor de idade.

# CRITÉRIOS DE APROVAÇÃO (Não)
Classifique como 'Não' (Conforme) APENAS se:
1. Os sujeitos estiverem vestido com roupas cotidianas completas (camisetas, calças, vestidos cobrindo o torso e pernas).
2. Não houver pessoas na imagem.
3. Todas as pessoas, mesmo que com roupas expositivas, são maiores de idade.

# FORMATO
[Sim/Não]; [Breve explicação técnica da classificação]
"""

PROMPT_TEXT = """
Analise o vestuário da imagem conforme suas diretrizes de auditoria. 
Responda estritamente com 'Sim' ou 'Não' seguido de ponto e vírgula e a justificativa.
"""
\end{lstlisting}

\begin{table}[htb]
	\IBGEtab{%
		\label{quad:prompt4}
		\caption{Performance com prompt 4}
	}{%
		\begin{tabular}{lccc}
			\toprule
			Métrica & Gemini 2.5 Flash  & Llama Scout  & Llama Maverick \\
			\midrule \midrule
			Acurácia & \(88,8\%\) & \(87,5\%\) & \(77,5\%\) \\
			\midrule 
			Precisão & \(81,6\%\) & \(85,7\%\) & \(92,3\%\) \\
			\midrule 
			Revocação & \(100,0\%\) & \(90,0\%\) & \(60,0\%\) \\
			\bottomrule
		\end{tabular}%
	}{%
		\fonte{Elaborada pelos autores.}%
	}
\end{table}

\section{Testes de temperatura}

Scout:
\begin{figure}[htb]
	\begin{center}
	\caption{Gráfico de performance do Llama Scout (usando prompt 3)}
    \label{fig:grafico_performance_scout}
	\includegraphics[width=0.8\textwidth]{USPSC-img/grafico_metricas_scout.png} \\
	\legend{Fonte: elaborado pelo autor}
	\end{center}	
\end{figure}

Maverick:
\begin{figure}[htb]
	\begin{center}
	\caption{Gráfico de performance do Llama Maverick (usando prompt 3)}
    \label{fig:grafico_performance_maverick}
	\includegraphics[width=0.8\textwidth]{USPSC-img/grafico_metricas_maverick.png} \\
	\legend{Fonte: elaborado pelo autor}
	\end{center}	
\end{figure}

Gemini:
\begin{figure}[htb]
	\begin{center}
	\caption{Gráfico de performance do Google Gemini 2.5 Flash (usando prompt 4)}
    \label{fig:grafico_performance_gemini}
	\includegraphics[width=0.8\textwidth]{USPSC-img/grafico_metricas_gemini.png} \\
	\legend{Fonte: elaborado pelo autor}
	\end{center}	
\end{figure}

\section{Testes de velocidade}

\begin{table}[htb]
	\IBGEtab{%
		\label{tab:tempo_execucao}
		\caption{Tempo de execução com melhor configuração de temperatura e prompt}
	}{%
		\begin{tabular}{lcccccc}
			\toprule
			Modelo & Temperatura & Prompt & Médio (s) & Mínimo (s) & Máximo (s) & Variância (s\(^2\))\\
			\midrule \midrule
			Scout & 0.5 & 3 & 1,36 & 0,98 & 1,78 & 0,03 \\
			\midrule 
			Maverick & 0 & 3 & 5,90 & 3,44 & 9,56 & 2,21 \\
			\midrule 
			2.5 Flash & 0 & 4 & 6,10 & 2,09 & 13,07 & 3,03 \\
			\bottomrule
		\end{tabular}%
	}{%
		\fonte{Elaborada pelo autor.}%
	}
\end{table}
% ---

% Capítulo 5
% ---
%!TEX root = tese/USPSC-TCC-modelo-EESC.tex
\chapter{Conclusão} \label{chap:conclusao}
Apresentar as conclusões correspondentes aos objetivos ou hipóteses propostos para o desenvolvimento do trabalho, podendo incluir sugestões para novas pesquisas.


% ---

% ----------------------------------------------------------
% ELEMENTOS PÓS-TEXTUAIS
% ----------------------------------------------------------
\postextual
% ----------------------------------------------------------

% -----------------------------------------------------------
% Referências bibliográficas
% ----------------------------------------------------------
\bibliography{USPSC-bib/USPSC-references}


% ----------------------------------------------------------
% Glossário
% ----------------------------------------------------------
%
% Consulte o manual da classe abntex2 para orientações sobre o glossário.
%
%\glossary

% ----------------------------------------------------------
% Apêndices
% ----------------------------------------------------------
%!TEX root = tese/main.tex
%% USPSC-Apendice.tex
% ---
% Inicia os apêndices
% ---

\begin{apendicesenv}
% Imprime uma página indicando o início dos apêndices
\partapendices

\chapter{Arquivo do manifesto} \label{cap:apendice_manifest}

Arquivo \texttt{manifest.json} usado para definir configurações gerais da extensão.
\lstinputlisting[
    language=json
]{../code/manifest.json}
Fonte: elaborado pelo autor

\end{apendicesenv}
% ---

% ----------------------------------------------------------
% Anexos
% ----------------------------------------------------------
% \include{USPSC-TA-PosTextual/USPSC-Anexos} % incluir anexos

%---------------------------------------------------------------------
% INDICE REMISSIVO
%--------------------------------------------------------------------
% %!TEX root = tese/main.tex
%% USPSC-IndicexRemissivos.tex
% ---
% Inicia os Índices Remissivos
% ---
%---------------------------------------------------------------------
% INDICE REMISSIVO
%--------------------------------------------------------------------
\phantompart
\printindex
%---------------------------------------------------------------------
 % incluir índice remissivo

%---------------------------------------------------------------------

\end{document}
