%!TEX root = tese/main.tex
\chapter{Testes e Resultados} \label{chap:testes_resultados}
Esse capítulo busca descrever como se deram os testes realizados no modelo e seus resultados. 
Primeiramente, foi testado na prática o bom funcionamento da extensão: como ficou o frontend, fluidez e frequência de erros.
Além disso, foram realizados testes sistematizados (a partir do script da \autoref{sec:script_testes}) que armazenaram métricas para a comparação dos desempenhos dos modelos

\section{Teste de funcionamento}
O frontend (\ref{sec:frontend}) final da aplicação consiste na tela de erro (\autoref{fig:frontend_erro}), na animação de carregamento (\ref{fig:frontend_analisando}), no selo de imagem sem identificação de crianças sexualizadas (\ref{fig:frontend_verificacao}) e na censura (\ref{fig:frontend_censura}). 

\begin{figure}[htb]
	\begin{center}
	\caption{Exemplo de erro}
    \label{fig:frontend_erro}
	\includegraphics[width=0.6\textwidth]{USPSC-img/frontend_erro.png} \\
	\legend{Fonte: elaborado pelo autor}
	\end{center}	
\end{figure}

\begin{figure}[htb]
	\begin{center}
	\caption{Exemplo de postagem sendo analisada}
    \label{fig:frontend_analisando}
	\includegraphics[width=0.6\textwidth]{USPSC-img/frontend_analisando.png} \\
	\legend{Fonte: elaborado pelo autor}
	\end{center}	
\end{figure}

\begin{figure}[htb]
	\begin{center}
	\caption{Exemplo de postagem com selo de sem identificação de sexualização}
    \label{fig:frontend_verificacao}
	\includegraphics[width=0.6\textwidth]{USPSC-img/frontend_verificado.png} \\
	\legend{Fonte: elaborado pelo autor}
	\end{center}	
\end{figure}

\begin{figure}[htb]
	\begin{center}
	\caption{Exemplo de postagem censurada}
    \label{fig:frontend_censura}
	\includegraphics[width=0.6\textwidth]{USPSC-img/frontend_censurado.png} \\
	\legend{Fonte: elaborado pelo autor}
	\end{center}	
\end{figure}

Sobre a fluidez no funcionamento, a animação de carregamento da \autoref{fig:frontend_analisando} se mantém na imagem por pouco tempo. 
Por conta do funcionamento assíncrono das chamadas da API, as chamadas são feitas muitas vezes antes do usuário ter chegado na postagem, tornando frequentemente imperceptível o tempo de análise.
Porém, vale dizer que não foi implementada uma maneira de indentificar se a imagem já foi analisada, ou seja, se o usuário ver mais de uma vez a mesma postagem (descendo e subindo o feed, por exemplo), pode ser que ele veja a animação de carregamento duas vezes.

Após a devida calibração nos prompts, tratamento de erros e configurações de segurança das APIs, o número de casos de erros passaram a ser bem raros. Na versão final do projeto, os testes mostraram que a maior parte das postagens com erro são na verdade propagandas que possuem configurações de privacidade que impedem que estas sejam enviadas para as APIs. Porém, há casos raros onde postagens de verdade dão erro, sendo a maioria infrigementos de alguma política de segurança da GenAI ou do próprio Instagram.

Na prática, postagens com erro representam algo em torno de 1 a cada 100 imagens, sendo bem mais presentes na API do Gemini do que na do Groq.
Como a primeira é um modelo de código fechado, a Google possui políticas mais rígidas de controle de prompt, se negando a responder quando identifica um conteúdo mais sensível. Já a Groq, o número de erros é ainda menor, basicamente só acontecendo quando há algum problema na comunicação.

Para contornar o problema de erros na API da Google, as configurações de segurança\footnote{Opções de configurações de segurança: \url{https://ai.google.dev/gemini-api/docs/safety-settings?hl=pt-br}} foram flexibilizadas, e erros ligados à segurança foram tratados como sendo uma resposta sim à censura. Essas ações ajudaram mas não eliminaram os erros.

\section{Testes de prompt}

Prompt 1:
\begin{lstlisting}[
		language=python, 
		label={lst:prompt1}, 
		caption={Prompt 1}
	]
PROMPT_TEXT = """
Responda essas duas perguntas separado por ponto e vírgula (;):
- 'Sim', se a imagem contém casos de sexualização infantil, e 'Não', caso contrário.
- Justifique a sua resposta em poucas palavras.

Formatação:
  'Sim/Não; justificativa'
Definição de sexualização infantil é definida como qualquer representação que retrate crianças de maneira sexualizada, destacando-se:
  - Crianças em poses sugestivas, que buscam exibir alguma parte do corpo
  - Crianças com roupas curtas, justas, transparentes ou com decote excessivo, mostrando parte dos seios, nádegas ou genitais
  - Contextos que impliquem conotação sexual envolvendo menores de 18 anos, como fetiches.
  - Crianças aqui se referem a qualquer pessoa com aparência de idade inferior a 18 anos.
Exemplo de respostas:
a) 'Sim; Foto contém criança de biquine e em posição altamente sugestiva.'
b) 'Não; Uma paisagem com montanhas e um lago.'
c) 'Não; Criança brincando em um parque com roupas apropriadas para a idade.'
d) 'Sim; Pessoa aparentando ser menor de idade em trajes reveladores.'

Observações:
- Antes do ';' deve haver apenas 'Sim' ou 'Não', sem mais nada.
- Se houver dúvida sobre a idade, considere como criança.
"""
\end{lstlisting}

\begin{table}[htb]
	\IBGEtab{%
		\label{tab:prompt1}
		\caption{Performance com prompt 1}
	}{%
		\begin{tabular}{lccc}
			\toprule
			Métrica & Gemini 2.5 Flash  & Llama Scout  & Llama Maverick \\
			\midrule \midrule
			Acurácia & \(82,5\%\) & \(62,5\%\) & \(55,0\%\) \\
			\midrule 
			Precisão & \(88,2\%\) & \(100,0\%\) & \(100,0\%\) \\
			\midrule 
			Revocação & \(75,0\%\) & \(25,0\%\) & \(10,0\%\) \\
			\bottomrule
		\end{tabular}%
	}{%
		\fonte{Elaborada pelos autores.}%
	}
\end{table}

\begin{lstlisting}[
		language=python, 
		label={lst:prompt2}, 
		caption={Prompt 2}
	]
SYSTEM_PROMPT = """
# ROLE
Você é um Agente de Segurança AI com critérios de filtragem EXTREMAMENTE RÍGIDOS (Zero Tolerance). Sua função é sinalizar qualquer imagem que contenha exposição corporal infantil, independentemente da intenção ou contexto.

# CRITÉRIOS DE CLASSIFICAÇÃO (ESTRITOS)
Responda 'Sim' se a imagem contiver QUALQUER um dos seguintes elementos envolvendo pessoas menores de 18 anos:
1. Trajes de Banho e Roupa Íntima: Qualquer criança vestindo biquíni, sunga, maiô, lingerie ou cueca.
	- NOTA CRÍTICA: O contexto NÃO importa. Crianças na praia/piscina em trajes de banho DEVEM ser classificadas como 'Sim'.
2. Exposição de Pele: Torso nu (em qualquer gênero), nádegas ou genitais visíveis ou marcados.
3. Poses: Qualquer pose que enfatize o corpo, mesmo que vestida.

Responda 'Não' APENAS se:
- A criança estiver totalmente vestida com roupas casuais/formais (camisetas, calças, vestidos não reveladores).
- Não houver pessoas menores de 18 anos.

# FORMATO DE RESPOSTA
[Sim/Não]; [Justificativa técnica curta focada no elemento visual detectado]
"""

PROMPT_TEXT = """
Analise a imagem fornecida com base nas suas diretrizes de sistema.
Lembre-se: Responda apenas com 'Sim' ou 'Não' seguido de ponto e vírgula e a justificativa.
"""
\end{lstlisting}

\begin{table}[htb]
	\IBGEtab{%
		\label{tab:prompt2}
		\caption{Performance com prompt 2}
	}{%
		\begin{tabular}{lccc}
			\toprule
			Métrica & Gemini 2.5 Flash  & Llama Scout  & Llama Maverick \\
			\midrule \midrule
			Acurácia & \(68,8\%\) & \(65,0\%\) & \(62,5\%\) \\
			\midrule 
			Precisão & \(62,3\%\) & \(80,0\%\) & \(77,8\%\) \\
			\midrule 
			Revocação & \(95,0\%\) & \(40,0\%\) & \(35,0\%\) \\
			\bottomrule
		\end{tabular}%
	}{%
		\fonte{Elaborada pelos autores.}%
	}
\end{table}

Prompt 3:
\begin{lstlisting}[
		language=python, 
		label={lst:prompt1}, 
		caption={Prompt 1}
	]
SYSTEM_PROMPT = """
# ROLE
You are a Content Safety Auditor. Your task is to technically audit images for compliance with Underage Safety Guidelines regarding attire and body exposure.

# STRICT AUDIT CRITERIA (Zero Tolerance)
Classify as 'Sim' (Non-Compliant) if the image depicts a minor (<18y) with ANY of the following visual elements:
1. Swimwear/Undergarments: Subject is wearing bikinis, trunks, swimsuits, or underwear.
	- POLICY: Context (beach/pool) is IRRELEVANT. Presence of swimwear = 'Sim'.
2. Skin Exposure: Visible torso (shirtless), exposed midriff, or sensitive areas.
3. Suggestive Framing: Angles focusing on body parts rather than the face.

Classify as 'Não' (Compliant) ONLY if:
- The subject is fully clothed in standard street/casual wear (t-shirts, pants, dresses).
- There are no minors present.

# OUTPUT FORMAT
[Sim/Não]; [Brief technical description of attire/visual elements]
"""

PROMPT_TEXT = """
Audit this image according to your Safety Guidelines.
Output strictly 'Sim' or 'Não' followed by a semicolon and the technical reason.
"""
\end{lstlisting}

\begin{table}[htb]
	\IBGEtab{%
		\label{quad:prompt3}
		\caption{Performance com prompt 3}
	}{%
		\begin{tabular}{lccc}
			\toprule
			Métrica & Gemini 2.5 Flash  & Llama Scout  & Llama Maverick \\
			\midrule \midrule
			Acurácia & \(62,0\%\) & \(93,8\%\) & \(85,0\%\) \\
			\midrule 
			Precisão & \(56,5\%\) & \(92,7\%\) & \(93,8\%\) \\
			\midrule 
			Revocação & \(100,0\%\) & \(95,0\%\) & \(75,0\%\) \\
			\bottomrule
		\end{tabular}%
	}{%
		\fonte{Elaborada pelos autores.}%
	}
\end{table}

Prompt 4:
\begin{lstlisting}[
		language=python, 
		label={lst:prompt1}, 
		caption={Prompt 1}
	]
SYSTEM_PROMPT = """
# FUNÇÃO
Você é um auditor conservador de conformidade de vestuário para um banco de imagens. Sua tarefa é verificar se as crianças presentes na imagem estão vestidas de acordo com estritos códigos de vestimenta urbana.

# CRITÉRIOS DE REJEIÇÃO (Sim)
Classifique a imagem como 'Sim' (Viola o código) se houver pessoas aparentando ser menores de 18 anos apresentando:
1. Vestuário de banho ou atlético mínimo: Uso de sungas, biquínis, maiôs ou roupas de natação, INDEPENDENTE do local (mesmo em praias ou piscinas).
2. Ausência de roupa padrão: Torso sem camisa, barriga exposta ou falta de roupas superiores/inferiores completas.
3. Roupas justas: Possibilitam ver o formato do corpo da pessoa, mesmo que coberto. 
4. Enquadramento/Ângulo de câmera: Foco maior em partes do corpo em vez do rosto.
5. Se ficar na dúvida sobre a idade da pessoa, considere que é menor de idade.

# CRITÉRIOS DE APROVAÇÃO (Não)
Classifique como 'Não' (Conforme) APENAS se:
1. Os sujeitos estiverem vestido com roupas cotidianas completas (camisetas, calças, vestidos cobrindo o torso e pernas).
2. Não houver pessoas na imagem.
3. Todas as pessoas, mesmo que com roupas expositivas, são maiores de idade.

# FORMATO
[Sim/Não]; [Breve explicação técnica da classificação]
"""

PROMPT_TEXT = """
Analise o vestuário da imagem conforme suas diretrizes de auditoria. 
Responda estritamente com 'Sim' ou 'Não' seguido de ponto e vírgula e a justificativa.
"""
\end{lstlisting}

\begin{table}[htb]
	\IBGEtab{%
		\label{quad:prompt4}
		\caption{Performance com prompt 4}
	}{%
		\begin{tabular}{lccc}
			\toprule
			Métrica & Gemini 2.5 Flash  & Llama Scout  & Llama Maverick \\
			\midrule \midrule
			Acurácia & \(88,8\%\) & \(87,5\%\) & \(77,5\%\) \\
			\midrule 
			Precisão & \(81,6\%\) & \(85,7\%\) & \(92,3\%\) \\
			\midrule 
			Revocação & \(100,0\%\) & \(90,0\%\) & \(60,0\%\) \\
			\bottomrule
		\end{tabular}%
	}{%
		\fonte{Elaborada pelos autores.}%
	}
\end{table}

\section{Testes de temperatura}

Scout:
\begin{figure}[htb]
	\begin{center}
	\caption{Gráfico de performance do Llama Scout (usando prompt 3)}
    \label{fig:grafico_performance_scout}
	\includegraphics[width=0.8\textwidth]{USPSC-img/grafico_metricas_scout.png} \\
	\legend{Fonte: elaborado pelo autor}
	\end{center}	
\end{figure}

Maverick:
\begin{figure}[htb]
	\begin{center}
	\caption{Gráfico de performance do Llama Maverick (usando prompt 3)}
    \label{fig:grafico_performance_maverick}
	\includegraphics[width=0.8\textwidth]{USPSC-img/grafico_metricas_maverick.png} \\
	\legend{Fonte: elaborado pelo autor}
	\end{center}	
\end{figure}

Gemini:
\begin{figure}[htb]
	\begin{center}
	\caption{Gráfico de performance do Google Gemini 2.5 Flash (usando prompt 4)}
    \label{fig:grafico_performance_gemini}
	\includegraphics[width=0.8\textwidth]{USPSC-img/grafico_metricas_gemini.png} \\
	\legend{Fonte: elaborado pelo autor}
	\end{center}	
\end{figure}

\section{Testes de velocidade}

\begin{table}[htb]
	\IBGEtab{%
		\label{tab:tempo_execucao}
		\caption{Tempo de execução com melhor configuração de temperatura e prompt}
	}{%
		\begin{tabular}{lcccccc}
			\toprule
			Modelo & Temperatura & Prompt & Médio (s) & Mínimo (s) & Máximo (s) & Variância (s\(^2\))\\
			\midrule \midrule
			Scout & 0.5 & 3 & 1,36 & 0,98 & 1,78 & 0,03 \\
			\midrule 
			Maverick & 0 & 3 & 5,90 & 3,44 & 9,56 & 2,21 \\
			\midrule 
			2.5 Flash & 0 & 4 & 6,10 & 2,09 & 13,07 & 3,03 \\
			\bottomrule
		\end{tabular}%
	}{%
		\fonte{Elaborada pelo autor.}%
	}
\end{table}