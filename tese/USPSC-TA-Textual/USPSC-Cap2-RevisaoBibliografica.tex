%!TEX root = tese/USPSC-TCC-modelo-EESC.tex
\chapter{Revisão bibliográfica} \label{chap:revisao_bibliografica}
Aqui é apresentada a fundamentação teórica do trabalho, 
por meio da análise de estudos prévios dos temas relevantes para o desenvolvimento do projeto e a justificativa das decisões metodológicas adotadas.

\section{Estudos sobre ``adultização'' nas redes sociais}
Atualmente, é amplamente estudado na literatura como as redes sociais impactam o crescimento de crianças.
Em muitos ambientes \textit{online}, não há restrições realmente eficientes ao acesso da criança em relação ao de um adulto:
elas podem, sem grandes dificuldades, utilizar ferramentas de buscas livremente, participar de interações sociais, e fazer \textit{download} e \textit{upload} de conteúdos, por exemplo.
Esse contato precoce com uma quantidade grande de informações pode impactar a psicologia e personalidade da criança,
levando a problemas de saúde mental, sentimentos de solidão, violentos ou de indiferença \cite{zheng2022}.

O estudo de \citeonline{yao2024}, realizado na China, analisou 2000 vídeos contendo crianças retirados de redes sociais como TikTok e Kwai, e outros 2000 contendo adultos.
Ao final, o estudo concluiu não só que ocorre a adultização das crianças, mas também a infantilização dos adultos. 
Nessa pesquisa, características identificadas como adultização foram, por exemplo, 
roupas reveladoras, e palavras e comportamentos sugestivos. 
\citeonline{yao2024} apontou como motivo para o fenômeno o rompimento do isolamento entre o virtual e o real, 
permitindo que as crianças participem do mundo adulto.

Esses e outros estudos \cite{orman2020, cabezas2022} mostram que de fato é um consenso na literatura que o consumo e exposição precoce no mundo digital pode impactar negativamente o desenvolvimento de um ser humano,
levando, em alguns casos, a uma ``adultização'' acelerada e descontrolada.  

\section{Revisão de modelos para detecção de sexualização}

\section{Revisão de modelos para identificação de crianças em imagens}

\section{Por que utilizar agente multimodal}
Por que não treinar um modelo do zero e por que não refinar um modelo já existente

Exemplificar casos de uso do mercado

\section{Agentes multimodais gratuitos disponíveis}

\section{Técnicas de \textit{Prompt Engineering}}
