%!TEX root = tese/main.tex
\chapter{Desenvolvimento}\label{cap_exemplos}

\section{Metodologia} \label{sec:metodologia}

No decorrer dessa seção serão definidas as ferramentas utilizadas para implementação do projeto em si e dos testes realizados.

\subsection{Manifest V3}
Extensões de navegador são pequenos módulos de \textit{software} que personalizam um navegador da web. 
Os navegadores normalmente permitem uma variedade de extensões, incluindo modificações na interface do usuário, gerenciamento de \textit{cookies}, bloqueio de anúncios e scripts e estilos personalizados de páginas da web.

O Manifest V3\footnote{Mudanças do Manifest V3: \url{https://developer.chrome.com/docs/extensions/develop/migrate/what-is-mv3?hl=pt-br}} (em referência ao arquivo de manifesto contido nas extensões) é a mais recente grande versão da API de extensões do Chrome
e visa modernizar a arquitetura de extensões e melhorar a segurança e o desempenho do navegador.
Ele adota APIs declarativas para diminuir a necessidade de acesso excessivamente amplo e permitir uma implementação mais eficiente, 
substitui páginas de fundo por ``Service Workers'' com recursos limitados para reduzir o uso de recursos e proíbe código hospedado remotamente. 

A escolha de uso do Manifest V3 se dá pois muitos navegadores suportam essa API.
Basicamente, todos os baseados no projeto Chromium são compatíveis, como Brave, Microsoft Edge, Opera, Vivaldi 
e, claro, o Google Chrome. Isso torna o uso dessa API bastante recomendada.

Para desenvolver uma extensão usando Manifest V3, é necessário criar um manifesto (\texttt{manifest.json}), 
arquivo que lista uma série de informações básicas sobre a extensão, e o navegador o usa realizar as configurações necessárias.
O \autoref{cap:apendice_manifest} contém o manifesto elaborado para o projeto, e inclui 
a versão do Manifest, o nome da extensão, a sua descrição, os caminhos para os códigos utilizados, as permissões necessárias, os arquivos das imagens utilizadas e outras informações.

Para esse extensão, os componentes mais importantes são o ``content script'' (\texttt{contentScript.js}) e o ``background service worker'' (\texttt{background.js}).
O content script é a parte da extensão que consegue conversar diretamente com o DOM (\textit{Document Object Model}) da página que o navegador está acessando,
que é basicamente a estrutura em si do site, ou seja, tudo o que o usuário vê.
O content script não é capaz de receber as permissões mais elevadas requisitadas no manifesto, 
para isso, ele precisa enviar uma mensagem para o background service worker, que possui esse privilégio e o retorna com a informação solicitada.
Toda essa aparente complicação a mais garante a segurança da extensão \cite{extensionvulnerabilities2010}.

\begin{figure}[htb]
	\begin{center}
	\caption{Arquitetura completa de uma extensão}
    \label{fig:extensions_arquitetura}
	\includegraphics[width=0.8\textwidth]{USPSC-img/extensions-architecture.png} \\
	\legend{Fonte: \url{https://youtu.be/TRwYaZPJ0h8?si=2gh3gnVSlcj5LRXF&t=194}}
	\end{center}	
\end{figure}



\subsection{Identificação de postagens do Instagram}
como é a estrutura de postagens do instagram

\subsection{Testes}
Criar conta fake, pesquisar se houve alguma mudança que hoje ta mais dificil

\section{Algoritmo} \label{sec:algoritmo}
Explicação geral do que acontec

\subsection{\textit{Backend}}

\subsection{\textit{Frontend}}

\subsection{Realização de testes}