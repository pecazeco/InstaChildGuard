%!TEX root = tese/main.tex
\chapter{Desenvolvimento}\label{cap_exemplos}

\section{Metodologia} \label{sec:metodologia}

No decorrer dessa seção serão definidas as ferramentas utilizadas para implementação do projeto em si e dos testes realizados.

\subsection{Manifest V3}
Extensões de navegador são pequenos módulos de \textit{software} que personalizam um navegador da web. 
Os navegadores normalmente permitem uma variedade de extensões, incluindo modificações na interface do usuário, gerenciamento de \textit{cookies}, bloqueio de anúncios e scripts e estilos personalizados de páginas da web.

O Manifest V3\footnote{Mudanças do Manifest V3: \url{https://developer.chrome.com/docs/extensions/develop/migrate/what-is-mv3?hl=pt-br}} (em referência ao arquivo de manifesto contido nas extensões) é a mais recente grande versão da API de extensões do Chrome
e visa modernizar a arquitetura de extensões e melhorar a segurança e o desempenho do navegador.
Ele adota APIs declarativas para diminuir a necessidade de acesso excessivamente amplo e permitir uma implementação mais eficiente, 
substitui páginas de fundo por ``Service Workers'' com recursos limitados para reduzir o uso de recursos e proíbe código hospedado remotamente. 

A escolha de uso do Manifest V3 se dá pois muitos navegadores suportam essa API.
Basicamente, todos os baseados no projeto Chromium são compatíveis, como Brave, Microsoft Edge, Opera, Vivaldi 
e, claro, o Google Chrome. Isso torna o uso dessa API bastante recomendada.

Para desenvolver uma extensão usando Manifest V3, é necessário criar um manifesto (\texttt{manifest.json}), 
arquivo que lista uma série de informações básicas sobre a extensão, e o navegador o usa realizar as configurações necessárias.
O \autoref{cap:apendice_manifest} contém o manifesto elaborado para o projeto, e inclui 
a versão do Manifest, o nome da extensão, a sua descrição, os caminhos para os códigos utilizados, as permissões necessárias, os arquivos das imagens utilizadas e outras informações.

Para esse extensão, os componentes mais importantes são o ``content script'' (\texttt{contentScript.js}) e o ``background service worker'' (\texttt{background.js}).
O content script é a parte da extensão que consegue conversar diretamente com o DOM (\textit{Document Object Model}) da página que o navegador está acessando,
que é basicamente a estrutura em si do site, ou seja, tudo o que o usuário vê.
O content script não é capaz de receber as permissões mais elevadas requisitadas no manifesto, 
para isso, ele precisa enviar uma mensagem para o background service worker, que possui esse privilégio e o retorna com a informação solicitada.
Toda essa aparente complicação a mais garante a segurança da extensão \cite{extensionvulnerabilities2010}.

\begin{figure}[htb]
	\begin{center}
	\caption{Arquitetura completa de uma extensão}
    \label{fig:extensions_arquitetura}
	\includegraphics[width=0.8\textwidth]{USPSC-img/extensions-architecture.png} \\
	\legend{Fonte: \url{https://youtu.be/TRwYaZPJ0h8?si=2gh3gnVSlcj5LRXF&t=194}}
	\end{center}	
\end{figure}

\subsection{Uso das APIs}

Um exemplo de chamada das APIs REST utilizadas por meio de cURL pode ser visto no \autoref{cap:apendice_doc_apis}.
As chamadas foram adaptadas para o uso em JavaScript, utilizando a função \texttt{fetch()}, que é nativa da linguagem e permite fazer requisições HTTP assíncronas.

\subsection{Testes}
Os passos da metodologia adotada para a realização dos testes, respectivos motivos das decisões tomadas e observações foram as seguintes:

\begin{enumerate}
	\item Criação de uma conta alternativa no Instagram
	\begin{itemize}
		\item Visando utilizar um algoritmo limpo, sem interferências de histórico ou preferências pessoais.
		\item Buscando não contaminar do perfil original do autor.
	\end{itemize}
	\item Diminuir o controle de conteúdo sensível nas configurações\footnote{Configuração de controle de conteúso sensível: \url{https://about.instagram.com/blog/announcements/introducing-sensitive-content-control}} da conta.
	\item Identificar e seguir perfis buscando por palavras chave relacionadas a crianças e a sexualização.
	\begin{itemize}
		\item Perfis que possuem na ``bio'' avisos falando que o perfil é monitorado por pais indicam que se trata possivelmente de uma criança.
		\item O Instagram restringe a busca por termos relacionados a sexualização de crianças diretamente (\autoref{fig:aviso_abuso}), logo, é necessário buscar os termos de sexualização e de crianças separadamente.
	\end{itemize}
	\item Navegar pelo feed e curtir e abrir comentários das postagens que tem algum tipo de sugestão sexual de crianças.
	\item Extrair (fazer download) de imagens que aparecerem no feed, avaliando manualmente se contém ou não sexualização infantil.
	\begin{itemize}
		\item Baixar metade sendo imagens que contenham sexualização de crianças e outra metade sendo imagens que não contenham, para ter resultados não enviesados.
	\end{itemize}
	\item Rodar scripts Python (\autoref{sec:script_testes}) desenvolvidos para avaliar as imagens baixadas utilizando o mesmo prompt utilizado na extensão.
	\begin{itemize}
		\item Utilizar scripts possibilita testes sistemáticos, garantindo que os testes serão realizados sempre sobre as mesmas imagens e de maneira mais ágil do que ficar navegando no feed e anotando os resultados.
		\item Foram feitos um script para cada API (Google e Groq), já que cada uma possui uma forma distinta de realizar a requisição.
	\end{itemize}
	\item Realizar adaptações no prompt do modelo buscando a maior acurácia possível.
	\item Repetir os últimos 2 passos até alcançar um resultado satisfatório. 
	\item Substituir o conjunto do melhor API + modelo + prompt no código da extensão JavaScript.
\end{enumerate}

\begin{figure}[htb]
	\begin{center}
	\caption{Aviso de abuso sexual}
    \label{fig:aviso_abuso}
	\includegraphics[width=0.5\textwidth]{USPSC-img/aviso-abuso-sexual.png} \\
	\legend{Fonte: Instagram}
	\end{center}	
\end{figure}

\section{Código} \label{sec:codigo}
Em um primeiro momento, a extensão proposta apenas analisa as postagens do Instagram dispostas no ``feed'', 
e somente as imagens. Não são avaliados os ``stories'' ou vídeos, porém, como o foco aqui é a prova de conceito, 
abranger esses casos seria uma adaptação de código, pois se o algoritmo funciona bem para a as imagens do feed, 
é natural também funcionar para os stories ou vídeos. 
Além disso, o código foi estruturado deixando aberturas para ser adaptado à outras abas do Instagram (reels, stories, explore), 
o que será explicado na \autoref{sec:backend}.

A explicação do funcionamento dos códigos desenvolvidos será dividida em três partes:
o \textit{backend} (onde a lógica do código e requisições acontecem), 
o \textit{frontend} (que se encarrega de montar o visual para o usuário),
e os scripts de teste utilizados para validar o funcionamento do \textit{backend}.

Vale enfatizar que, como já disponibilizado na \autoref{sec:escopo_do_projeto}, todo o código-fonte completo do projeto está disponível em um repositório público no GitHub.

\subsection{\textit{Backend}} \label{sec:backend}

\subsection{\textit{Frontend}}

\subsection{Script para testes} \label{sec:script_testes}