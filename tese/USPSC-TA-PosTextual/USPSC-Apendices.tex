%!TEX root = tese/main.tex
%% USPSC-Apendice.tex
% ---
% Inicia os apêndices
% ---

\begin{apendicesenv}
% Imprime uma página indicando o início dos apêndices
\partapendices

\chapter{Arquivo do manifesto} \label{cap:apendice_manifest}

Arquivo \texttt{manifest.json} usado para definir configurações gerais da extensão.
\lstinputlisting[
    language=json
]{../code/manifest.json}
Fonte: elaborado pelo autor

\chapter{Documentação de chamada das APIs} \label{cap:apendice_doc_apis}

Pela documentação oficial da API do Google uma chamada pode ser feita como exemplificada pelo \autoref{lst:gemini_api_doc}.

\begin{lstlisting}[
		language=sh, 
		label={lst:gemini_api_doc}, 
		caption={Exemplo de envio de imagem à API do Gemini 2.5 Flash}
	]
curl "https://generativelanguage.googleapis.com/v1beta/models/gemini-2.5-flash:generateContent" \
-H "x-goog-api-key: $GEMINI_API_KEY" \
-H 'Content-Type: application/json' \
-X POST \
-d '{
    "contents": [{
        "parts":[
            {
                "inline_data": {
                    "mime_type":"'"$MIME_TYPE"'",
                    "data": "'"$IMAGE_B64"'"
                }
            },
            {"text": "Caption this image."}
        ]
    }]
}' 2> /dev/null
\end{lstlisting}
Fonte: \url{https://ai.google.dev/gemini-api/docs/image-understanding?hl=pt-br#rest}

Já pela documentação da Groq, a chamada pode ser feita conforme o \autoref{lst:groq_api_doc}.

\begin{lstlisting}[
		language=sh, 
		label={lst:groq_api_doc}, 
		caption={Exemplo de envio de imagem à API do Groq}
	]
curl "https://api.groq.com/openai/v1/chat/completions" \
  -X POST \
  -H "Content-Type: application/json" \
  -H "Authorization: Bearer ${GROQ_API_KEY}" \
  -d '{
         "messages": [
           {
             "role": "user",
             "content": [
               {
                 "type": "text",
                 "text": "What'\''s in this image?"
               },
               {
                 "type": "image_url",
                 "image_url": {
                   "url": "'https://upload.wikimedia.org/wikipedia/commons/f/f2/LPU-v1-die.jpg'"
                 }
               }
             ]
           }
         ],
         "model": "meta-llama/llama-4-scout-17b-16e-instruct",
         "temperature": 1,
         "max_completion_tokens": 1024,
         "top_p": 1,
         "stream": false,
         "stop": null
       }'
\end{lstlisting}
Fonte: \url{https://console.groq.com/docs/vision}

\chapter{Arquivo CSS com os estilos} \label{cap:apendice_css}

Arquivo \texttt{styles.css} usado para definir o estilo do frontend.
\lstinputlisting[
    language=css
]{../code/styles.css}
Fonte: elaborado pelo autor

\end{apendicesenv}
% ---