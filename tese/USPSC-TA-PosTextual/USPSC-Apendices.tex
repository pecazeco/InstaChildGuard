%!TEX root = tese/main.tex
%% USPSC-Apendice.tex
% ---
% Inicia os apêndices
% ---

\begin{apendicesenv}
% Imprime uma página indicando o início dos apêndices
\partapendices

\chapter{Arquivo do manifesto} \label{cap:apendice_manifest}

Arquivo \texttt{manifest.json} usado para definir configurações gerais da extensão.
\lstinputlisting[
    language=json
]{../code/manifest.json}
Fonte: elaborado pelo autor

\chapter{Documentação de chamada das APIs} \label{cap:apendice_doc_apis}

Pela documentação oficial das APIs do Google e do Groq, as chamadas por cURL podem ser feitas como exemplificado no \autoref{lst:gemini_curl_doc} e \ref{lst:groq_curl_doc}.

\begin{lstlisting}[
		language=sh, 
		label={lst:gemini_curl_doc}, 
		caption={Exemplo de envio de imagem à API do Gemini 2.5 Flash}
	]
curl "https://generativelanguage.googleapis.com/v1beta/models/gemini-2.5-flash:generateContent" \
-H "x-goog-api-key: $GEMINI_API_KEY" \
-H 'Content-Type: application/json' \
-X POST \
-d '{
    "contents": [{
        "parts":[
            {
                "inline_data": {
                    "mime_type":"'"$MIME_TYPE"'",
                    "data": "'"$IMAGE_B64"'"
                }
            },
            {"text": "Caption this image."}
        ]
    }]
}' 2> /dev/null
\end{lstlisting}
Fonte: \url{https://ai.google.dev/gemini-api/docs/image-understanding?hl=pt-br#rest}

\begin{lstlisting}[
		language=sh, 
		label={lst:groq_curl_doc}, 
		caption={Exemplo de envio de imagem à API do Groq}
	]
curl "https://api.groq.com/openai/v1/chat/completions" \
  -X POST \
  -H "Content-Type: application/json" \
  -H "Authorization: Bearer ${GROQ_API_KEY}" \
  -d '{
         "messages": [
           {
             "role": "user",
             "content": [
               {
                 "type": "text",
                 "text": "What'\''s in this image?"
               },
               {
                 "type": "image_url",
                 "image_url": {
                   "url": "'https://upload.wikimedia.org/wikipedia/commons/f/f2/LPU-v1-die.jpg'"
                 }
               }
             ]
           }
         ],
         "model": "meta-llama/llama-4-scout-17b-16e-instruct",
         "temperature": 1,
         "max_completion_tokens": 1024,
         "top_p": 1,
         "stream": false,
         "stop": null
       }'
\end{lstlisting}
Fonte: \url{https://console.groq.com/docs/vision}

Já as chamadas usando Python podem ser observadas no \autoref{lst:doc_gemini_python} e \ref{lst:doc_groq_python} 

\begin{lstlisting}[
  language=python,
  label={lst:doc_gemini_python},
  caption={Chamada à API do Gemini 2.5 Flash usando Python}
]
from google import genai
from google.genai import types

with open('path/to/small-sample.jpg', 'rb') as f:
    image_bytes = f.read()

client = genai.Client()
response = client.models.generate_content(
  model='gemini-2.5-flash',
  contents=[
    types.Part.from_bytes(
      data=image_bytes,
      mime_type='image/jpeg',
    ),
    'Caption this image.'
  ]
)

print(response.text)
\end{lstlisting}
Fonte: \url{https://ai.google.dev/gemini-api/docs/image-understanding?hl=pt-br#rest}

\begin{lstlisting}[
  language=python,
  label={lst:doc_groq_python},
  caption={Chamada à API do Groq usando Python}
]
from groq import Groq
import base64
import os

# Function to encode the image
def encode_image(image_path):
  with open(image_path, "rb") as image_file:
    return base64.b64encode(image_file.read()).decode('utf-8')

image_path = "sf.jpg"
base64_image = encode_image(image_path)
client = Groq(api_key=os.environ.get("GROQ_API_KEY"))
chat_completion = client.chat.completions.create(
    messages=[
        {
            "role": "user",
            "content": [
                {"type": "text", "text": "What's in this image?"},
                {
                    "type": "image_url",
                    "image_url": {
                        "url": f"data:image/jpeg;base64,{base64_image}",
                    },
                },
            ],
        }
    ],
    model="meta-llama/llama-4-scout-17b-16e-instruct",
)

print(chat_completion.choices[0].message.content)
\end{lstlisting}
Fonte: \url{https://console.groq.com/docs/vision}

\chapter{Arquivo CSS com os estilos} \label{cap:apendice_css}

Arquivo \texttt{styles.css} usado para definir o estilo do frontend.
\lstinputlisting[
    language=css
]{../code/styles.css}
Fonte: elaborado pelo autor

\end{apendicesenv}
% ---